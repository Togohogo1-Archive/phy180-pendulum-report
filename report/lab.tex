\documentclass[12pt]{article}

\usepackage{graphicx}
\usepackage[sorting=none]{biblatex}
\usepackage[margin=1in]{geometry}
\usepackage[colorlinks=true]{hyperref}
\usepackage{amsmath}
\usepackage{amssymb}
\usepackage[format=plain, labelfont=it, font=footnotesize, labelsep=period]{caption}
\usepackage{caption}
\usepackage{subcaption}

\addbibresource{references.bib}

\title{Pendulum Lab Report II}
\author{Kevin (Zerui) Wang}
\date{\today}

\begin{document}

\pagenumbering{gobble}

\maketitle
\newpage

\pagenumbering{arabic}

\section{Introduction}
{\color{blue}The purpose of this lab report is to evaluate the accuracy of various theoretical models used to predict the behaviour of a simple pendulum.}

The first part of this lab report focuses on the relationship the period and release angle of a simple pendulum. After results were collected and analyzed, an improved version of the pendulum was created to collect angular data in order to determine the Q factor.

{\color{blue}Lastly, further analysis was done on the Q factor to try to determine a trend in relation to the pendulum length}

\section{Background} \label{Background}
{\color{blue}A simple pendulum can be defined as a weight suspended below a pivot point to which it swings back and forth freely along a plane that contains the pivot point via a flexible string. The length of the pendulum is defined as the distance from the pivot to the center of mass of the weight hanging from the pendulum. Usually, the weight of the string is negligible in comparison to the pendulum mass.} For a simple pendulum of length $L$ experiencing a downwards force due to gravity $g$, the period $T$, for a release angle $\theta$, is given by the following equation \cite{the-simple-pendulum}:
\begin{equation} \label{eq:l-over-g}
    T = 2\pi \sqrt{\frac{L}{g}}
\end{equation}

{\color{blue}Note that Equation \ref{eq:l-over-g} does not depend on $\theta$ because it represents the behaviour of an ideal pendulum.} More realistically, the behavior of a pendulum assuming no energy loss due to friction can be modelled with a second order differential equation of $\theta$ with respect to $t$:
\begin{equation} \label{eq:diffeq-pendulum}
    \frac{d^2\theta}{dt^2} + \frac{g}{L}\sin{\theta} = 0
\end{equation}
{\color{blue}which cannot be solved in terms of elementary functions \cite{no-elementary-fns}. However, if small angle approximation is assumed, where $\sin\theta \approx \theta$, the differential equation can be rewritten and solved to yield $\theta(t)$ in terms of a sine wave with period of Equation \ref{eq:l-over-g}. For $|\theta| \lesssim 20^{\circ}$, small angle approximation (and therefore simple harmonic motion) holds \cite{the-simple-pendulum}.} Additionally, the the mass of the pendulum weight does not show up in any of the equations, implying that pendulum period does not depend on mass for the previous models.

Another way to model a pendulum realistically is to incorporate dampening from air resistance and friction within the string fibres, given by the following equation \cite{damped-oscillations}:
\begin{equation} \label{eq:damped-harmonic-oscillator}
    \theta(t) = \theta_0 e^{-{t/\tau}} \cos\left(2\pi\frac{t}{T} + \phi_0\right)
\end{equation}
where $\theta_0$ is the initial release angle in radians, $T$ is the pendulum's period, $\phi_0$ is the angular phase shift, and $\tau$ {\color{blue} is the decay constant that governs all factors that results in the damped harmonic motion. However, it can be inferred from this model that $T$ is constant, implying the presence of small angle approximation.}

{\color{blue}
Additionally, the amplitude-time graph of the pendulum can be modelled by removing the cosine term from Equation \ref{eq:damped-harmonic-oscillator}:
\begin{equation} \label{eq:amplitude-function}
    A(t) = \theta_0 e^{-{t/\tau}}
\end{equation}}

In order to quantify the dampening effect of a pendulum, its Q factor may be used. The Q factor measures how damped an oscillator is and is defined as follows. \cite{pnp-physics}:
\begin{equation} \label{eq:q-factor-formula}
    Q = \pi\frac{\tau}{T}
\end{equation}

The Q factor also represents the the number of oscillations for the pendulum's amplitude to decay to $e^{-\pi}$, or $\approx 4\%$ of its original amplitude. {\color{blue}However, it suffices to count $Q/n$ swings where the amplitude decays to $e^{-\pi/n}\,\%$ of its original. Qualitatively, when the Q factor is large, the pendulum comes to rest slower. When the Q factor is small, the pendulum comes to rest quicker.}

\newpage

\section{Period and Release Angle}

\subsection{Experimental Setup}
The initial setup for the pendulum was made by first attaching a protractor on a desk. The pendulum, made by tying a piece of cotton string around a stainless steel quick link {\color{blue} that was aligned to match the $90^{\circ}$ mark on the pendulum. Extra tape was also used to reinforce the pivot point, which was made to be collinear to the $0^{\circ}$ and $180^{\circ}$ markings.} The length of the pendulum was measured using measuring tape. An image of the experimental setup is shown below:

\begin{figure}[!hptb]
    \centering
    \includegraphics[width=0.5\textwidth]{../figures/exp_setup1.jpg}
    \caption{\centering Picture of experimental setup for period vs. release angle data collection}
    \label{fig:figure 1}
\end{figure}

\newpage

\subsection{Data}
The data collected for the period vs. release angle graph is shown in the plot below:

\begin{figure}[!hptb]
    \centering
    \includegraphics[width=\textwidth]{../figures/period_vs_release_angle.png}
    \caption{\centering Period plotted against release angle of the setup in Figure \ref{fig:figure 1}}
    \label{fig:figure 2}
\end{figure}

All data collected for this graph were done without the need for tracking software. The uncertainty for the protractor (measured by eye) was taken to be the smallest increment, $0.5^{\circ}$ converted to radians, and the period uncertainty was taken to be the average human reaction time, $0.25\,\text{s}$ \cite{reaction-time} {\color{blue} since it is greater than the uncertainty of a stopwatch. The period itself was measured by recording the pendulum for 3 swings and dividing the total time by 3. The maximum height was taken as the reference point because there would be less certainty in the pendulum's position (lower speed).}

\subsection{Analysis}
{\color{blue}
talk about what the b value and the c value means

assymetry

experimentally zero

concluse that period depends on release angle

comment on the r sq value alongside the residuals

write out the actual uncertianty values for t0, b, c
}

According to Section \ref{Background}, if $-20^{\circ} \lesssim \theta \lesssim 20^{\circ}$, then a linear line of best fit would be expected, which matches the trends generated by the graph.

Since the data collected contained angles ranging from $ 80^{\circ} \leq \theta \leq 80^{\circ}$, a quadratic power series fit modelled by the equation:
\begin{equation} \label{eq:power series}
    T_0(1 + B\theta_0 + C\theta_0^2)
\end{equation}
where $T_0$ represents the period, $\theta_0$ the release angle and $B$ and $C$ some arbitrary constants would pass through more error bars and leave smaller overall residuals than a linear fit. Thus, it can be concluded that the pendulum's period depends on amplitude, which becomes more apparent for larger angles.

Lastly, the residuals shown on the graph do not suggest much asymmetry regarding the pendulum as the negative release angles appear symmetrical when compared to positive release angles. However, a single-stringed pendulum tends to spin in an elliptical fashion when released, affecting accurate measurements.

\newpage

\section{Finding the Q Factor} \label{Finding the Q Factor}

\subsection{Experimental Setup}
In order to correct for potential asymmetry and elliptical orbit, a degree of freedom was removed from the pendulum by threading a string through the stainless steel quick link and fastening the 2 strings in 2 locations to form the shape of a ``V''. Then pendulum would then swing back and forth in the plane perpendicular to the ``V'', and the tendency for each string to balance out the load prevents it from swinging in any other direction. Additionally, the pendulum was kept the same length as the previous setup (perpendicular distance from threading location to top of desk), and extra precautions were taken to make sure the pendulum was symmetrical along the vertical axis. Below is a setup of what was described above:
\begin{figure}[!hptb]
    \centering
    \includegraphics[width=0.5\textwidth]{../figures/exp_setup2.jpg}
    \caption{\centering Improved version of the pendulum used to determine Q factor}
    \label{fig:figure 3}
\end{figure}

\newpage

\subsection{Data}
Due to the double-string contraption, the camera must be positioned in a place where the chance of capturing both strings in the same frame is minimized.

A protractor was not included in the experimental setup this time because Tracker \cite{tracker} was used to measure the angles. First, data was collected such that the maximum amplitude of the pendulum could be plotted against time:

\begin{figure}[!hptb]
    \centering
    \includegraphics[width=\textwidth]{../figures/max_amplitude_vs_time.png}
    \caption{\centering Graph of maximum amplitude vs. time}
    \label{fig:figure 4}
\end{figure}

The uncertainty for time was taken to be $1/f$ where $f$ represents the frame rate of the video imported into Tracker, namely $120\,\text{fps}$. The uncertainty for angle was calculated this time for each individual maximum by taking half the range of the sweeping area that the string made in Tracker due to video blur.

\subsection{Analysis}
The exponential fit used to create a curve of best fit for the data was taken from Equation \ref{eq:damped-harmonic-oscillator}, removing the periodic motion portion. The equation for the fit is equal to:
\begin{equation}
    1.21e^{-{t}/68}
\end{equation}

With this information and also knowing the period ($1.67\,\text{s}$) of the pendulum when the release angle is $1.2064\,\text{rad}$, the $Q$ factor can be calculated. The uncertainty of the $Q$ factor is taken to be the largest percent uncertainty of $\tau$ and $T$.

$\tau$ was calculated to have an uncertainty of 1, which is less than the percent uncertainty of $T$, taken to be $0.3/1.67 \approx 15\%$. Thus, the value of $Q$ can be calculated to be $130 \pm 20$.

However, Graph \ref{fig:figure 4} can also be used to find the Q factor by counting the number of oscillations. Since the total amplitude does not decrease down to $e^-\pi \%$ of the original amplitude, it suffices to calculate a value for $Q/5$. The closes data points above and below $\theta_0e^{-{pi/5}}$ correspond to the $24^\text{th}$ and $25^\text{th}$ swing. Multiplying these values by 5 and taking half the difference (range) to be the uncertainty gives another Q factor of $123 \pm 2$.

%%%%%%%%%%%%%%%%%%%%%%%%%%%
%%%%%%%%%%%%%%%%%%%%%%%%%%%
%%%%%%%%%%%%%%%%%%%%%%%%%%%
%%%%%%%%%%%%%%%%%%%%%%%%%%%
%%%%%%%%%%%%%%%%%%%%%%%%%%%
% THIS IS ALL NEW
%%%%%%%%%%%%%%%%%%%%%%%%%%%
%%%%%%%%%%%%%%%%%%%%%%%%%%%
%%%%%%%%%%%%%%%%%%%%%%%%%%%
%%%%%%%%%%%%%%%%%%%%%%%%%%%
%%%%%%%%%%%%%%%%%%%%%%%%%%%

{\color{blue}

\section{Period and Length} \label{Period and Length}

\subsection{Experimental Setup}
A third version of the pendulum was made to collect data for 9 different lengths ranging from around $10\,$cm to $50\,$cm, as shown below:

\begin{figure}[!hptb]
    \centering
    \begin{subfigure}{0.49\textwidth}
        \centering
        \includegraphics[width=\textwidth]{../figures/exp_setup3_angle.png}
    \end{subfigure}
    \hfill
    \begin{subfigure}{0.49\textwidth}
        \centering
        \includegraphics[width=\textwidth]{../figures/exp_setup3_front.png}
    \end{subfigure}
    \caption{Side and front view of version 3 of the pendulum}
    \label{fig:figure 5}
\end{figure}

The setup used here doesn't deviate much from the one used in Section \ref{Finding the Q Factor}, other than the fact that a real protractor was used in place of Tracker's virtual one, lego was used to prevent the potential risk of the string slipping under the tape and the angle formed by the "V" shape was kept to be around $60^\circ$ for all lengths recorded. The last point is especially important so that any dependent variable under investigation can only depend on one independent variable, the length of the pendulum.

\subsection{Data}

\subsection{Analysis}

\section{Q Factor and Length}
The same setup from Section \ref{Period and Length} was used to calculate the Q factors.

Q factors were calculated using the python program because the counting method is inaccurate in general

\subsection{Data}

\subsection{Analysis}

\section{Conclusion}
simple pendulum equation holds fo small angles

amplitude time graph also holds for small angles

the small angles stuff can be concluded based on the period vs release angle graph

q factor graph nobody knows what it is, I came up with a plausible function based off of intuition

}

\newpage

\printbibliography

\end{document}
