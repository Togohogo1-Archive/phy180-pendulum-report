\documentclass[12pt]{article}

\usepackage[]{biblatex}
\usepackage[margin=1in]{geometry}
\usepackage[]{hyperref}

\addbibresource{references.bib}

\title{Pendulum Lab Report 1}
\author{Kevin (Zerui) Wang}
\date{\today}

\begin{document}

\pagenumbering{gobble}

\maketitle

\pagenumbering{arabic}

\section{Introduction}
The first part of this lab report focuses on the relationship the period and release angle of a simple pendulum. After results were collected and analyzed, an improved version of the pendulum was created to collect angular data in order to determine the Q factor.

\section*{Background} \label{Background}
- period vs releae angle for basic pendulum would be linear, i.e. period should not depend on release angle \\
- when does the $2\pi\sqrt{\frac{L}{g}}$ fail to hold? \\
- introduce small angle approximation, find somewhere that mentions what degrees small angle approximation holds \\
-


\section{Period and Release Angle}



\subsection{Experimental Setup}
\subsection{Data}
\subsection{Analysis}
- Provide a clear conclusion about whether your pendulum's period depends on ampli- tude. If you do find some dependence, you should clearly indicate what range (if any) of amplitudes are `small enough' that the value of C can be ignored. Be clear as to what criteria you used to make a `small enough' judgment. \ref*{Background}

\section{Finding the Q Factor}
- [objective] fix for asymmetry

\subsection{Experimental Setup}
\subsection{Data}
\subsection{Comparing Q Factors}
\subsection{Analysis}
- this is gonna be an exponential fit because the overall residuals are smaller than when the fit is linear [include a figure containing both]
- Describe how you took this data (specifically including the impact of the Q factor on your choices)

\section{Notes}


\begin{itemize}
    \item period is dependent on angle \cite*{what-is-yeast}
    \item If you find that the period does depend on the angle, and you find that your Q factor is not huge (in the 1000s), then you will need to be careful about how you measure the period. (i.e. conclusions of Q factor based on angle?)
    \item compare the ``3'' ways of calculating the Q factor
\end{itemize}


\textbf{Procedure}\\
for version 2 of the pendulum, I got rid of a swinging motion by attaching 2 strings

asymmetry can be tested by taking residuals?

talk about uncertainties (make a section for it?)

\begin{equation}
    \theta(t) = 0.86 e^{-\tau/30} \cos\left(2 \pi \frac{t}{1.43} - 0.08\right)
\end{equation}

\section*{Conclusion}
\nocite{*}

\newpage

\printbibliography

\end{document}
