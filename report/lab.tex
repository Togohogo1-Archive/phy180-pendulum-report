\documentclass[12pt]{article}

\usepackage[margin=1in]{geometry}

\title{Pendulum Lab Report}
\author{Kevin (Zerui) Wang}
\date{\today}

\begin{document}

\pagenumbering{gobble}

\maketitle

\pagenumbering{arabic}

\section*{Introduction and Objectives}
The first part of this lab report focuses on the relationship the period and release angle of a simple pendulum. After results were collected and analyzed, an improved version of the pendulum was created to collect angular data in order to determine the Q factor.


\begin{itemize}
    \item period is dependent on angle
    \item If you find that the period does depend on the angle, and you find that your Q factor is not huge (in the 1000s), then you will need to be careful about how you measure the period. (i.e. conclusions of Q factor based on angle?)
    \item compare the ``3'' ways of calculating the Q factor
\end{itemize}

\section*{Background}

\section*{Procedure}
for version 2 of the pendulum, I got rid of a swinging motion by attaching 2 strings

asymmetry can be tested by taking residuals?

talk about uncertainties (make a section for it?)

\begin{equation}
    \theta(t) = 0.86 e^{-\tau/30} \cos\left(2 \pi \frac{t}{1.43} - 0.08\right)
\end{equation}
\section*{Conclusion}

\end{document}
