\documentclass[main.tex]{subfiles}
\begin{document}
\section{Q Factor Derivation}
the start of the appendixe
lefishe

Q factor is total oscillations until amplitude decays to $e^{-\pi}\theta_0$.

\begin{align*}
    \text{total oscillations} = Q \times \frac{\text{energy lost}}{\text{1 oscillation}} &= \text{total energy lost} \\
    \implies \quad Q &= \frac{E_T}{E_{osc}}
\end{align*}

Assuming that the energy lost per oscillation is constant \\

total energy loss to decay is equal to the difference in gravitational potential energy from $\theta_0$ to $e^{-\pi}\theta_0$ since from the following diagram:
<labelled pendulum diagram here or can just say determine geometrically> \\

also redefine $e^{-\pi}\theta_0 = \theta_f$

\begin{align*}
    E_T &= mgL(1-\cos\theta_0) - mgL(1-\cos\theta_f)\\
    &= mgL(\cos\theta_f-\cos\theta_0)
\end{align*}

previous equation implies that $E_T \propto L$

\begin{equation}
    E_T = c_1 L
\end{equation}

Now energy loss per oscillation - the forces that are doing the work are friction and air resistance

taking the pendulum diagram again, where $\theta_1$ = angle of pendulum after 1 oscillation

\begin{equation}
    mgL(1-cos\theta_0) = mgL(1-cos\theta_1) = W_{f} + W_{d}
\end{equation}

more assumptions -> from differential equation to solve for harmonic oscillator we know that $F_f \propto \frac{d\theta}{dt} = \omega$

since $\omega = \frac{2\pi}{T}$ and also Equation \ref{eq:l-over-g}, $\omega \propto \frac{1}{\sqrt{L}} \implies F_f \propto \frac{1}{\sqrt{L}}$

Work = $F\times d$ and $d = L(\theta_0 - \theta_1) = \text{arc length}$, therefore

\begin{align}
    W_f &= F_f L(\theta_0 - \theta_1) \\
    &\propto \frac{L}{\sqrt{L}} \\
    &= c_2\sqrt{L}
\end{align}

For the air resistance, we know that $F_d = \frac{1}{2}\rho C_d A v^2$. However $v^2$ is simply $(\omega L)^2 = \left(\frac{2\pi}{T(L)} L\right)^2 = k_1L$. $\rho$ and $C_d$ doesn't change for the pendulum. $A$ changes with length - this is because as the string gets longer, the total surface area increases. With the increase in length, the drag force on the string along is porportional to its length, but the surface area of the bob must also be taken into account, resulting in the area exposed to air for pendulum to be equal to $(Lw + A_{bob})$, where $w$ is the width of the string, which does not change. Thus,

\begin{equation}
    F_d = \frac{1}{2}\rho C_d (Lw + A_{bob}) k_1L
\end{equation}

Then the work done is represnted by
\begin{align*}
    W_d &= F_d L(\theta_0 - \theta_1) \\
    &= c_4L^2(L+c_3)
\end{align*}

Thus, the final equation for Q factor is represented by the equation
\begin{align}
    Q(L) &= \frac{c_1L}{c_2\sqrt{L} + c_4L^2(L+c_3)} \\
    &= \frac{d_1L}{d_2\sqrt{L} + L^2(L+d_3)}
\end{align}

where $d_1$, $d_2$, and $d_3$ are new constants after dividing the top and bottom by $c_4$.
\end{document}
